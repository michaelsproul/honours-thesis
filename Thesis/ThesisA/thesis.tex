\documentclass[]{unswthesis}

\usepackage{hyperref}
\usepackage{graphicx}
\usepackage{float}
\usepackage{enumerate}
\usepackage{amsmath, amssymb}
\usepackage{mathtools}
\usepackage[utf8]{inputenc}
\usepackage{url}
\usepackage{multicol}
\usepackage[nottoc,numbib]{tocbibind}
\usepackage{proof}
% \usepackage[scale=2]{ccicons}

% Disable hideous fluorescent links
\hypersetup{hidelinks=true}

%%% Class options:

%  undergrad (default)
%  hdr

%  11pt (default)
%  12pt

%  final (default)
%  draft

%  oneside (default for hdr)
%  twoside (default for undergrad)


%% Thesis details
\thesistitle{Computer-verified proof of type soundness for the Linear Language with Locations, $L^3$}
\thesisschool{School of Computer Science and Engineering}
\thesisauthor{Michael Alexander Sproul}
\thesisZid{z3484357}
\thesistopic{} % TODO: Find out what this is meant to be?
\thesisdegree{Bachelor of Science (Honours)}
\thesisdate{October 2015}
\thesissupervisor{Dr. Ben Lippmeier}

%% My own LaTeX macros, definitions, etc.
\let\oldemptyset\emptyset
\let\emptyset\varnothing
\newcommand{\case}{\text{ case }}
\newcommand{\of}{\text{of }}
\newcommand{\yields}{\multimap}
\newcommand{\steps}{\Rightarrow}
\newcommand{\lquine}{\left\ulcorner}
\newcommand{\rquine}{\right\urcorner}
\newcommand{\capa}{\text{cap}}
\newcommand{\ptr}{\text{ptr }}
\newcommand{\rgnUL}{$\lambda^\text{rgnUL}$\text{ }}

\begin{document}

%% pages in the ``frontmatter'' section have roman numeral page number
\frontmatter  
\maketitle

% \chapter*{Abstract}
\label{abstract}

Write this last.

% \chapter*{Acknowledgements}
\label{ack}

Write this at some point.

% Do I need these?
% \chapter*{Abbreviations}\label{abbr}
\begin{description}
\item[BE] Bachelor of Engineering
\item[EE\&T] School of Electrical Engineering and Telecommunication
\item[\LaTeX] A document preparation computer program
\item[PhD] Doctor of Philosophy
\end{description}


\tableofcontents
%\listoffigures
%\listoftables

%% pages in the ``mainmatter'' section have arabic page numbers and chapters are numbered
\mainmatter

% \chapter{Introduction}\label{ch:intro}

Having a set of clear requirements to their thesis is important to student
finalising their BE, or other, degree.  Such requirements are both in
relation to the physical appearance of the thesis, as well as the writing
style and organisation.  The present document tries to concisely state the
theses requirements while appearing in layout and structure as a thesis
itself.

Chapter~\ref{ch:background} explains the background for this document.
Chapter~\ref{ch:style} states the style and submission related requirements
to theses submitted at the school.
Chapter~\ref{ch:content} explains content related requirements to theses.
Chapter~\ref{ch:eval} evaluates the thesis requirements template.  Finally,
Chapter~\ref{ch:conclusion} draws up conclusions and suggest ways to
further improve the thesis requirements template.


\chapter{Introduction}
\label{ch:intro}

Computer systems form an integral part of modern society, both in the form of personal devices and critical infrastructure. Ensuring the correct operation of computer hardware and software is therefore required. One emerging technique for the construction of robust software systems is the use of mathematical formalisations and proofs of correctness. In this paradigm, desirable properties of the software can be proven true using a computer-based \textit{proof assistant}, which itself relies only on a minimal amount of trusted code. For software formalisation and verification to be truly effective, the objects under consideration must have precise mathematical models associated with them. Typically these models are created based on the \textit{semantics} (meaning) of the programming language that the software is written in. Unfortunately for the would-be software verifier, most popular programming languages lack formal semantics and are therefore not amenable to verification techniques. The focus of this thesis is the computer-based formalisation of language semantics for a specific language (\textit{The Linear Language with Locations} -- $L^3$), as a pre-requisite for further verification of software written in this language.

\section{Operational Semantics}

\section{Type Systems and Type Safety}

REF Wright and Felleisen (1992).

\section{Logic, Type Theory and the Coq proof assistant}

\section{Verification Aims}

The $L^3$ language was specified in a 2001 (2005? 2007?) paper by Ahmed et al (REF). The paper includes a hand-written proof of type soundness for $L^3$ core, spanning 8 pages.

Include motivation for linear/uniqueness typing here.

% Background.
\chapter{Background and Previous Work}
\label{ch:intro}

% Overall picture (verified all the way down, focus on low-level languages).
% Focus on "normal" semantics and type systems.
% Notes about TAL

If software verification is to propagate through modern software stacks, verification of components at different layers is a necessity. Proofs about programs written in high-level languages offer only superficial assurances if during compilation down to executable machine code* the program is corrupted by an unverified transformation. Verification of entire computing systems including compilers, operating systems and file-systems is a huge undertaking however, so we restrict our attention here to the type systems of low-level languages.

Historically, low-level \textit{systems software} has been written primarily in the C programming language, which was originally designed without a formal semantics. Attempts to assign semantics to C have been quite successful, and have resulted in numerous impressive verification projects -- notably the CompCert verified compiler (REF) and the seL4 micro-kernel (REF). In this work we consider type systems for languages that could potentially act as verification-friendly successors to the domains where C currently excels (operating systems, language runtimes, embedded devices). Our core hypothesis is that \textit{uniqueness types} and the destructive updates they enable should simplify the verification of systems software.

% FIXME: make this a footnote.
*We don't consider other methods of computation in the present paper, e.g. logic gates and qubits.

\section{Linear and Affine Typing}

Linear, affine and uniqueness typing are closely-related features of type systems that enforce rules about the number of times values may be used and referenced. These restrictions are motivated by several desirable features that can be obtained by enforcing them. The Clean programming language (REF) uses uniqueness typing to ensure that values in memory have at most one reference to them, thus enabling \textit{destructive updates} whilst preserving referential transparency. The Rust programming language (REF) uses uniqueness typing to track and free heap-allocated memory, thus allowing it to achieve memory safety without garbage collection. This makes it suitable for writing systems software where a garbage collector isn't available (like a garbage collector itself, or an operating system).

Linear and affine type systems are based on linear and affine logic respectively (REF Girard and Grishin), via the standard mapping from logic to type systems (see previous section?). In classical and intuitionistic logic there are deduction rules equivalent to the following, called \textit{Contraction} and \textit{Weakening}.

\begin{eqnarray*}
\infer[\text{Contraction}]{\Gamma, A \vdash B}{
	\Gamma, A, A \vdash B
}
\qquad
\infer[\text{Weakening}]{\Gamma, A \vdash B}{
    \Gamma \vdash B
}
\end{eqnarray*}

Contraction allows duplicate assumptions to be discarded, whilst Weakening allows a non-vital extra assumption to be introduced from nowhere. Linear logic bans the use of both rules, such that every assumption is used \textit{exactly once}. Affine logic on the other hand bans only the use of Contraction, which results in the requirement that every assumption be used \textit{at most once}. When transformed into typing rules, Contraction and Weakening take the form:

\begin{eqnarray*}
\infer[\text{Contraction}]{\Gamma, x : A \vdash u[x/y, x/z] :: B}{
	\Gamma, y : A, z : A \vdash u :: B
}
\qquad
\infer[\text{Weakening}]{\Gamma, x : A \vdash y :: B}{
    \Gamma\vdash y :: B
}
\end{eqnarray*}

(Note that the use of substitutions in the rule for Contraction is required so that no variable appears more than once in a typing context $\Gamma$).

As in linear and affine logic, linear type theory prevents the use of Contraction and Weakening, whilst affine type theory prevents just the use of Contraction. Our previous observations about the number of times an assumption is used now translate into observations about the number of times variables are used.

Without Contraction, \textit{variables can only appear once in a term}. The dual substitution of $x$ for $y$ and $x$ for $z$ allows a term containing one $y$ and one $z$ to become a term containing two occurrences of $x$. None of the other rules of intuitionistic type theory allow this (REF Wadler).

Similarly, without Weakening, \textit{all variables in the context must be used in the term}. This follows from the fact that Weakening introduces an unused variable to the context and no other rule of intuitionistic type theory allows this.

From these two observations we can conclude that linear type theory requires all variables to be used \textit{exactly once} in terms, whilst affine type theory requires all variables to be used \textit{at most once} in terms.

It's also worth noting that linear and affine type theory don't entirely ban Contraction and Weakening -- their use is allowed for certain \textit{non-linear} values and types. This is quite practical in the context of programming language implementation as it allows trivially copyable values like integers to be easily duplicated. The mechanism employed by Wadler to selectively allow Contraction and Weakening involves differentiating between linear and non-linear assumptions, and adding type and value-level bang (!) operators to make types and values non-linear. The function for duplication of non-linear values (of type $!A$) is expressed as follows in such a system :

\begin{eqnarray*}
\varnothing \vdash \lambda \langle x' \rangle . \case x' \of !x \rightarrow \langle !x, !x\rangle :\text{ } !A \yields (!A \otimes !A)
\end{eqnarray*}

See Wadler 1993 for full notational details.

\section{Uniqueness Typing}

Although linear and affine type theory capture the essence of uniquely referenced values, they are insufficient to describe the concept of \textit{uniqueness} as it appears in languages like Clean. In his 2008 paper, de Vries (REF) notes that terms of a unique type are \textit{guaranteed to never have been shared}, which is sufficient to guarantee a unique pointer at runtime. In contrast, terms of linear (or affine) type are \textit{guaranteed not to be shared in the future}, which is insufficient to guarantee a unique pointer.

Without implying equivalence of the two concepts, let $\alpha^\bullet$ denote the type of both linear and unique values, for any base type $\alpha$ (such as \texttt{Int}). Similarly, allow the notation $\alpha^\times$ to stand for non-linear and non-unique types.

\begin{eqnarray*}
\lambda x \cdot x : & \alpha^\times \rightarrow \alpha^\bullet & \text{(linear dereliction)}\\
\lambda x \cdot x : & \alpha^\bullet \rightarrow \alpha^\times & \text{(uniqueness removal)}
\end{eqnarray*}

The distinctness of linearity and uniqueness is further highlighted by the \textit{dereliction} rule present in some versions of linear type theory, and the rule that we'll refer to as \textit{uniqueness removal} present in Clean's type system. As noted by de Vries (REF de Vries PhD thesis), the presence of the dereliction rule allows a non-linear value to be transformed into a linear one, thus breaking any alleged equivalence of linearity and uniqueness (by definition, a non-unique value should not suddenly become unique). Conversely, intuition about uniqueness suggests that treating a unique value as non-unique should be allowed - hence, uniqueness removal.

Note that dereliction need not form a part of type systems based on linear logic, and that it is absent from Wadler's presentations (REF 1991, 1993). Further note that if uniqueness is to be exploited to make garbage collection unnecessary -- as in the case of Rust -- then the uniqueness removal rule is undesirable as it prevents values from having a unique owner.

Due to the non-equivalence of linearity and uniqueness, de Vries constructed a distinct set of semantics and typing rules to model Clean's type system (REF de Vries).

One key component of his approach is the use of the \textit{kind} (type of types) system to track uniqueness and non-uniqueness. As in Haskell, de Vries' uniqueness system includes a kind for data (\texttt{*}) which is the kind of all inhabited types. In addition, there is a uniqueness kind $\mathcal{U}$ inhabited by two types $\bullet$ and $\times$ representing uniqueness and non-uniqueness respectively. A third kind, $\mathcal{T}$ is the kind of base types (like \texttt{Int}). These kinds are brought together by a type constructor $\texttt{Attr} ::_k \mathcal{T} \rightarrow \mathcal{U} \rightarrow *$ which applies a uniqueness attribute to a base type to form a type that is inhabited (e.g. \texttt{Attr} $\bullet$ \texttt{Int} or $\texttt{Int}^\bullet$ for the type of uniquely referenced integers).

The other main technique employed by de Vries' model is the use of arbitrary boolean expressions as uniqueness attributes, with $\bullet$ as true and $\times$ as false. Clean's type system allows uniqueness polymorphism, which results in constraint relationships between uniqueness variables, which are represented in de Vries' system as simple boolean expressions that can be handled by a standard unification algorithm. FIXME: Does this need to be here?

Importantly, de Vries work includes the only known mechanical verification of a type system similar to Clean's. The formalisation uses the Coq proof assistant, and the \textit{locally nameless} (REF) approach to variable naming that will be discussed in (FIXME: SOME SECTION).

% Linear + affine logic and control over contraction and weakening. DONE.

% Uniqueness typing as an alternative (mention dereliction, non guarantees). DONE.

% de Vries formalisation of Clean's uniqueness typing using attributes.

\section{The Linear Language with Locations, $L^3$}

Explain $L^3$ and linear capabilities here. Including: runtime erasure of capabilities.

\pagebreak

\section{Systems of Capabilities}

\subsection{Cyclone}

Several systems extend and generalise the capability-based approach employed in $L^3$. Fluet, Morrisett and Ahmed followed up their paper on $L^3$ with a region-based system that borrows many ideas from $L^3$, called \rgnUL (REF). Notably, it makes use of linear capabilities to provide safe access to \textit{dynamic regions}, which are first-class abstractions for the allocation of memory. Dynamic regions extend simpler lexical regions by allowing regions to exist independent of lexical scopes. Accompanying the \rgnUL paper is a mechanised proof of type soundness using the Twelf proof assistant (REF).

The same authors are also responsible for the Cyclone project (REF), which extends the C programming language with regions and uniqueness typing in order to achieve safe memory management without garbage collection or manual intervention. The \rgnUL calculus models many of Cyclone's features, and there exists a translation from Cyclone to \rgnUL via an intermediate language $F^\text{RGN}$ which makes use of a generalised ST monad (REF Monadic regions). No mechanised proofs of correctness for this work exist, although an earlier proof of type soundness for Cyclone (REF formal type soundness for Cyclone's region system) is structured in a way that looks amenable to mechanised verification. On their webpage (REF), the creators of Cyclone note that work on the project has stopped, with many of the ideas living on in Rust. Future formalisations of Rust can hopefully make use of this work.

\subsection{Pottier's type-and-capability system with hidden state}

A formalisation for a system very similar to $L^3$ does exist, in a 2013 article by Fran\c{c}ois Pottier (REF). Pottier's system, SSPHS, uses affine capabilities in the style of $L^3$, but adds polymorphism and support for \textit{hidden state}. Hidden state allows an object to completely conceal mutable internal state from its clients. Pottier gives a memory manager as an example where such a feature is useful -- clients care only about the memory allocated or de-allocated, and not about internal data-structures modified in the process of (de)allocation. Hidden state is realised via a typing rule called the \textit{anti-frame rule}, which makes terms with hidden state subtypes of the type sans hidden state. The result is a type system for a low-level language supporting ownership and strong updates.

FIXME: Could probably expand on the frame and anti-frame rules here.

Pottier's formalisation is done within the Coq proof assistant and makes use of de Bruijn indices for variable binding (a pre-cursor to his \texttt{dblib} library, discussed in FIXME). The formalisation consists of 20,000 lines of Coq source and follows the syntactic approach to proving type soundness via progress and preservation (REF Wright). Pottier notes that the formalisation took around 6 months to complete.

\subsection{Mezzo and high-level systems}

Together with Protzenko, Pottier is also responsible for the Mezzo programming language and its associated Coq formalisation (REF). Mezzo differs from SSPHS and \rgnUL in that it is designed to be high-level and expressive. Its system of ownership is based around linear \textit{permissions}, which allow programmers to design diverse usage \textit{protocols} for functions and data. Mezzo's model of concurrency leverages ownership to guarantee that well-typed programs do not contain data-races, a property that is also formalised in Coq.

The prototypical compiler for Mezzo uses untyped OCaml as its target language and as such requires garbage collection at runtime. Further, due to OCaml's lack of parallelism, concurrent and race-free Mezzo programs are unable to take advantage of multiple cores. One can imagine further work to compile Mezzo to a low-level language with similar semantics, in order to take advantage of its full feature set.

Mezzo's Coq formalisation consists of 14,000 lines of code and makes use of a 2000 line library called \texttt{dblib} for handling de Bruijn indices. Like the proof for SSPHS, it uses progress and preservation to prove type soundness.

% rgnURAL as a basis for Cyclone. DONE.

% Francois Pottier's low-level thing. DONE.

% Mezzo. Strong updates, etc. Kind of DONE.

\section{Typed assembly languages and trustworthy compilers}

Strong updates can be used to model the storage of type-distinct values in a single register through-out program execution. As such, low-level calculi like $L^3$ and \rgnUL are conceptually linked to \textit{typed assembly languages} (TALs), which extend regular assembly languages with type annotations. Well-typed TAL programs typically guarantee memory safety given an axiomatisation of a machine architecture. In the TALx86 (REF) system, blocks are annotated with pre-conditions that place requirements on the types of registers. A semantic model for typed assembly languages has recently appeared (REF Semantic Foundations for TALs), including a Twelf formalisation. Denotational semantics are used in the proof of type soundness, in contrast to the operational semantics and syntactic soundness proofs of previously cited works.

More broadly it is worth noting the contribution of the CompCert (REF) project to program verification. Through a series of semantics-preserving translations through intermediate languages, CompCert compiles a variant of C to multiple assembly languages. CompCert is programmed and verified in Coq. Verification of programs written in a low-level linearly-typed language could use parts of CompCert, perhaps with a language like $L^3$ or SSPHS as an intermediate language.

Another take on the typed assembly language concept, is Bedrock from Adam Chlipala's research group (REF). FIXME: Keep going here (Bedrock seems pretty massive).

% TALx86. DONE.
% CompCERT. DONE.
% Bedrock.

\section{Summary of Previous Work}



\chapter{Proposal}

Proposal intro here.

\section{Variable naming and binding}

One problem that arises frequently in the formalisation of language semantics is that of \textit{capture-avoiding substitution}. Substitution operations, whereby a value is substituted for a variable in a term, form the core computational component of the operational semantics in many languages. In the simply-typed (and untyped) lambda calculus, the $\beta$-rule uses substitution (denoted $e[v/x]$) to describe the semantics of function application:
% FIXME: formatting, long right arrow and space.
\begin{eqnarray*}
(\lambda x : \tau. e) v \Rightarrow_\beta e[v/x]
\end{eqnarray*}

The problem of \textit{variable capture}, which we wish to avoid, is demonstrated by the following example:
% FIXME: formatting
\begin{eqnarray*}
(\lambda x. \lambda y. x + y) y \Rightarrow_\beta (\lambda y. y + y)
\end{eqnarray*}

Here the parameter $y$ is a free variable acting as a place-holder for a value in the environment. After substitution however, the $y$ replacing $x$ in the abstraction body $x + y$ becomes bound due to the name collision between the free $y$ and the binder $y$. Intuitively, drastically altering the meaning of terms during substitution is something we would like to avoid.

One way to avoid variable capture is to forbid the substitution of any terms containing free variables. In such a system, free variables like $y$ are never considered values and as such cannot be used in (variable capturing) substitutions. This is the approach taken by \textit{Software Foundations} (REF) in formalisations of the simply-typed lambda calculus and its variants. A further consequence of this approach is that globally-shared integers or strings for variable names are sufficient to guarantee soundness. Although it's tempting to embrace this approach for its simplifying properties, it doesn't accurately capture the behaviour of common functional languages like Haskell and ML, which perform substitutions whilst avoiding variable capture. (might need a stronger argument here).

In our Coq formalisation of $L^3$ we would therefore like to include \textit{capture avoiding substitution} as part of the definitions of variable names and substitution operations. For this we consider three main approaches from the literature which all exploit the observation that the exact names of bound variables are insignificant at the level of language formalisation. In other words, although the names of variables may hold meaning for the authors of programs, they do not impact the meaning of the programs themselves.

\subsection{Higher-order Abstract Syntax}

\subsection{de Bruijn indices}

\subsection{The Locally Nameless approach}

\section{Coq Formalisation of $L^3$}

\section{Summary of Proposal}



%\chapter{Background}
\label{ch:background}

Etc.

%\include{proposal}
%\chapter{Style and Submission Requirements}\label{ch:style}

Requirements for other parts of the thesis work can be found on the school
web-pages~\cite{Noo05}.  The requirements below are for the written thesis
only.

\section{Format}
The following format specifications must be adhered to for your thesis
(the \LaTeX\ template available from the school ensures this):
\begin{enumerate}
\item The thesis must be printed on \emph{A4 size paper}.
\item The thesis must be typed or prepared using a \emph{word-processor}.
\begin{itemize}
\item For Undergraduate theses, you are encouraged to use both sides
  of the paper.
\item For Higher Degree Research theses, your submitted thesis must be
   printed single-sided.
\end{itemize}
\item \emph{Margins} on all sides must be no less than \unit[25]{mm} (before
binding).
\item \emph{1.5 line spacing} (about \unit[8]{mm} per line) must be used.
\item All sheets must be \emph{numbered}. The main body of the thesis must be
numbered consecutively from beginning to end.  Other sections must either
be included or have their own logical numbering system.
\item The \emph{title page} must contain the following information:
\begin{enumerate}
\item University and School names.
\item Title of Thesis/Project.
\item Topic Number (if applicable).
\item Name of Author and student ID.
\item The degree the thesis is submitted for.
\item Submission date (month and year).
\item Supervisor's name (for undergraduate theses).
\end{enumerate}
\item After the body of the thesis, the thesis \emph{must} contain a
  Bibliography or References list as appropriate.

Authors should confer with their supervisors and School about the
style of their bibliography, as this varies between disciplines.
\end{enumerate}

\section{Other physical appearance}
Other requirements to the physical appearance of your theses are:
\begin{enumerate}
\item The report must be \emph{spiral bound} (at your own cost).
\item Formulas and other items difficult to type may be \emph{neatly
hand-written}
in \emph{permanent} black ink.
\item \emph{Graphs, diagrams and photographs} should be inserted as close as
possible to their \emph{first reference} in the text. Rotated
graphs etc are to be arranged so as to be conveniently read, with the
bottom edge to the outside of the page.
\emph{Graphs and diagrams must be legible!}
\item \emph{Photographs} must be permanently attached to sheets at least along
their left edge. Double sided adhesive may be
used to attach photographs. Photographs printed on A4 size lightweight
paper may be bound directly into the thesis.
\item \emph{Computer programs} and \emph{engineering drawings} should be bound into the
thesis, usually in an appendix.
\item \emph{Floppy diskettes/CD} may be attached to the back cover of the thesis
folder using self adhesive tape or in a secure
pocket.
\end{enumerate}

\section{Submission}

Finally, here are some requirements to the submission procedure. 

\begin{enumerate}
\item The \emph{author} of the thesis is \emph{responsible} for the preparation of the
thesis before the deadline, proofreading the
typescript and having corrections made as necessary.
\item All students must submit a \emph{thesis summary sheet} with their thesis
report. This summary sheet is designed to assist
in determining the overall input by students into the thesis work. Please
note that a separate summary sheet must
be submitted by individual student, even if part of a group submitting a
group thesis.
The guidelines for completing
the summary sheet and the summary sheet form can be downloaded from the
School Office Website.
\item \emph{Two copies} of each thesis/group thesis report must be submitted.
\item Students doing a \emph{Group Thesis} are required to write and hand in
\emph{individual reports}.  The reports should be
clearly distinguishable, and appropriately cross referenced to each other.
The common work overlapping between the reports should be clearly
identified.
\item For undergraduate theses, there is a \emph{page limit} of 100 pages for the main body of the thesis.
\end{enumerate}



\chapter{Content Requirements}\label{ch:content}

Students should consult the literature (e.g.~\cite{Sid99,StrWhi79,Coo64,GRS14})
and other resources for material on how to write a good
thesis.  The present document is only a very brief introduction as to what
is expected.

\nocite{NieLeh03,HasLehKwo05}

\section{Structure}
Most theses are structured very much like the present document.
The main part of the thesis can be structured in many different ways,
however, but must contain: a \emph{problem definition};
\emph{theory} and \emph{considerations} on how to solve the problem;
a description of the \emph{solution method} (dimensioning, construction,
etc.);
presentation of \emph{results} (measurements, simulations, etc.);
a \emph{discussion} of the results (validity, deviations, comparison
with previous solutions, etc.); and finally the \emph{conclusions}.

\section{Style of writing}

\begin{enumerate}

\item Audience:
The thesis must be addressed to engineers at the same level as the
student but without the special knowledge gained during the thesis work.
Such a third-person must be able to reconstruct the results on the basis
of the thesis alone.

\item
Every used concept/symbol/abbreviation which is not widely know must be \emph{defined}.
The wording should be \emph{short} and \emph{concise}.  
For an undergraduate thesis, a suitable length
is 40--70 pages (plus appendices).
Readable(!) \emph{figures} and \emph{graphs} enhances comprehensibility.

\item Units.
\emph{SI units} must be used.
\end{enumerate}

\section{Documentation}

\begin{enumerate}
\item
The work must be well documented; i.e. enclosed must be the \emph{complete
schematics} of designed electronic circuits/test set-ups and/or a
\emph{program listing}, and/or etc.
Documentation of \emph{simulation results} and/or \emph{measurement
results} likewise.
\item References:
For every declaration/equation/method/etc., which is not widely known,
a \emph{reference to the literature} must be given (or a `proof' if it is
the authors own work).
In case material is copied verbatim, quotes must be used.
This is also the case when referring to partners
work in the case of a Group Thesis.

\item Plagiarism:
Failure to give proper references to the literature is \emph{plagiarism}.
Plagiarism is considered serious offence and severe penalties may apply.

\end{enumerate}


%\chapter{Evaluation}\label{ch:eval}

This chapter is mainly provided for the purpose of showing a typical thesis
structure.  There are no more thesis requirements described.

\section{Results}

The result of this work is the present document, being both a \LaTeX\
template and a thesis requirement specification.

\section{Discussion}

The Dual function of this document somewhat de-emphasises the primary
purpose of the document, namely the thesis requirements.  It would be
better, if these could be stated on a few concise pages (cf Appendix
1, p\pageref{app1}).

%\chapter{Conclusion}\label{ch:conclusion}

A thesis requirements/template document has been created.  This serves the
dual purposes of giving students specific requirements to their theses ---
both style and content related --- while providing a typical thesis
structure in a \LaTeX\ template.

\section{Future Work}

Extract the requirements from the template in order to have very concise
requirements.


%% chapters in the ``backmatter'' section do not have chapter numbering
%% text in the ``backmatter'' is single spaced
\backmatter
\bibliographystyle{alpha}
\bibliography{pubs}

%\chapter{Appendix 1}\label{app1}

This section contains the options for the UNSW thesis class; and
layout specifications used by this thesis.

\section{Options}

The standard thesis class options provided are:

\qquad
\begin{tabular}{rl}
undergrad & default \\
hdr & \\[2ex]
11pt & default\\
12pt &\\[2ex]
oneside & default for HDR theses\\
twoside & default for undergraduate theses\\[2ex]
draft & (prints DRAFT on title page and in footer and omits pictures)\\
final & default\\[2ex]
doublespacing & default\\
singlespacing & (only for use while drafting)
\end{tabular}

\section{Margins}

The standard margins for theses in Engineering are as follows:

\qquad
\begin{tabular}{|l|r|r|}
\hline
 & U'grad & HDR\\\hline
{\verb+\oddsidemargin+} & \unit[40]{mm} & \unit[40]{mm}\\
{\verb+\evensidemargin+} & \unit[25]{mm} & \unit[20]{mm}\\
{\verb+\topmargin+} & \unit[25]{mm} & \unit[30]{mm}\\
{\verb+\headheight+} & \unit[40]{mm} & \unit[40]{mm}\\
{\verb+\headsep+} & \unit[40]{mm} & \unit[40]{mm}\\
{\verb+\footskip+} & \unit[15]{mm} & \unit[15]{mm}\\
{\verb+\botmargin+} & \unit[20]{mm} & \unit[20]{mm}\\
\hline
\end{tabular}

\section{Page Headers}

\subsection{Undergraduate Theses}
For undergraduate theses, the page header for odd numbers pages in the
body of the document is:

\quad\fbox{\parbox{.95\textwidth}{Author's Name\hfill \emph{The title of the thesis}}}

and on even pages is:

\quad\fbox{\parbox{.95\textwidth}{\emph{The title of the thesis}\hfill Author's Name}}

These headers are printed on all mainmatter and backmatter pages,
including the first page of chapters or appendices.

\subsection{Higher Degree Research Theses}
For postgraduate theses, the page header for the body of the document is:

\quad\fbox{\parbox{.95\textwidth}{\emph{The title of the chapter or appendix}}}

This header is printed on all mainmatter and backmatter pages,
except for the first page of chapters or appendices.

\section{Page Footers}

For all theses, the page footer consists of a centred page number.  
In the frontmatter, the page number is in roman numerals.  
In the mainmatter and backmatter sections, the page number is in arabic numerals.
Page numbers restart from 1 at the start of the mainmatter section.  

If the \textbf{draft} document option has been selected, then a ``Draft'' message is also inserted into the footer, as in:

\quad\fbox{\parbox{.95\textwidth}{\hfill 14\hfill\hbox to 0pt{\hss\textbf{Draft:} \today}}}

or, on even numbered pages in two-sided mode:

\quad\fbox{\parbox{.95\textwidth}{\leavevmode\hbox to 0pt{\textbf{Draft:} \today\hss}\hfill 14\hfill\mbox{}}}

\section{Double Spacing}
Double spacing (actualy 1.5 spacing) is used for the mainmatter section, except for
footnotes and the text for figures and table.

Single spacing is used in the frontmatter and backmatter sections.

If it is necessary to switch between single-spacing and double-spacing, the commands \verb+\ssp+ and \verb+\dsp+ can be used; or there is a \verb+sspacing+ environment to invoke single spacing and a \verb+spacing+ environment to invoke double spacing if double spacing is used for the document (otherwise it leaves it in single spacing).  Note that switching to single spacing should only be done within the spirit of this thesis class, otherwise it may breach UNSW thesis format guidelines.

\section{Files}

This description and sample of the UNSW Thesis \LaTeX\ class consists of a number of files:

\quad\begin{tabular}{rl}
unswthesis.cls & the thesis class file itself\\[2ex]
crest.pdf & the UNSW coat of arms, used by \verb+pdflatex+ \\
crest.eps & the UNSW coat of arms, used by \verb+latex+ + \verb+dvips+ \\[2ex]
dissertation-sheet.tex & formal information required by HDR theses\\[2ex]
pubs.bib & reference details for use in the bibliography\\[2ex]
sample-thesis.tex & the main file for the thesis
\end{tabular}

The file sample-thesis.tex is the main file for the current document (in use,
its name should be changed to something more meaningful).  It presents
the structure of the thesis, then includes a number of separate files
for the various content sections.  While including separate files is
not essential (it could all be in one file), using multiple files is
useful for organising complex work.

This sample thesis is typical of many theses; however, new authors should
consult with their supervisors and exercise judgement.

The included files used by this sample thesis are:

\quad\begin{tabular}[t]{r}
definitions.tex \\
abstract.tex \\
acknowledgements.tex \\
abbreviations.tex \\
introduction.tex \\
background.tex
\end{tabular}
\quad\begin{tabular}[t]{r}
mywork.tex \\
evaluation.tex \\
conclusion.tex \\
appendix1.tex \\
appendix2.tex 
\end{tabular}

These are typical; however the concepts and names
(and obviously content) of the files making up the matter of the
thesis will differ between theses.

%\chapter{Appendix 2}\label{app2}

This section contains scads of supplimentary data.

\section{Data}

Heaps and heaps and heaps and heaps and heaps and heaps of data.
Heaps and heaps and heaps and heaps and heaps and heaps of data.
Heaps and heaps and heaps and heaps and heaps and heaps of data.
Heaps and heaps and heaps and heaps and heaps and heaps of data.
Heaps and heaps and heaps and heaps and heaps and heaps of data.

Heaps and heaps and heaps and heaps and heaps and heaps of data.
Heaps and heaps and heaps and heaps and heaps and heaps of data.
Heaps and heaps and heaps and heaps and heaps and heaps of data.
Heaps and heaps and heaps and heaps and heaps and heaps of data.
Heaps and heaps and heaps and heaps and heaps and heaps of data.

Heaps and heaps and heaps and heaps and heaps and heaps of data.
Heaps and heaps and heaps and heaps and heaps and heaps of data.
Heaps and heaps and heaps and heaps and heaps and heaps of data.
Heaps and heaps and heaps and heaps and heaps and heaps of data.
Heaps and heaps and heaps and heaps and heaps and heaps of data.

Heaps and heaps and heaps and heaps and heaps and heaps of data.
Heaps and heaps and heaps and heaps and heaps and heaps of data.
Heaps and heaps and heaps and heaps and heaps and heaps of data.
Heaps and heaps and heaps and heaps and heaps and heaps of data.
Heaps and heaps and heaps and heaps and heaps and heaps of data.

Heaps and heaps and heaps and heaps and heaps and heaps of data.
Heaps and heaps and heaps and heaps and heaps and heaps of data.
Heaps and heaps and heaps and heaps and heaps and heaps of data.
Heaps and heaps and heaps and heaps and heaps and heaps of data.
Heaps and heaps and heaps and heaps and heaps and heaps of data.

Heaps and heaps and heaps and heaps and heaps and heaps of data.
Heaps and heaps and heaps and heaps and heaps and heaps of data.
Heaps and heaps and heaps and heaps and heaps and heaps of data.
Heaps and heaps and heaps and heaps and heaps and heaps of data.
Heaps and heaps and heaps and heaps and heaps and heaps of data.

Heaps and heaps and heaps and heaps and heaps and heaps of data.
Heaps and heaps and heaps and heaps and heaps and heaps of data.
Heaps and heaps and heaps and heaps and heaps and heaps of data.
Heaps and heaps and heaps and heaps and heaps and heaps of data.
Heaps and heaps and heaps and heaps and heaps and heaps of data.

Heaps and heaps and heaps and heaps and heaps and heaps of data.
Heaps and heaps and heaps and heaps and heaps and heaps of data.
Heaps and heaps and heaps and heaps and heaps and heaps of data.
Heaps and heaps and heaps and heaps and heaps and heaps of data.
Heaps and heaps and heaps and heaps and heaps and heaps of data.

Heaps and heaps and heaps and heaps and heaps and heaps of data.
Heaps and heaps and heaps and heaps and heaps and heaps of data.
Heaps and heaps and heaps and heaps and heaps and heaps of data.
Heaps and heaps and heaps and heaps and heaps and heaps of data.
Heaps and heaps and heaps and heaps and heaps and heaps of data.

Heaps and heaps and heaps and heaps and heaps and heaps of data.
Heaps and heaps and heaps and heaps and heaps and heaps of data.
Heaps and heaps and heaps and heaps and heaps and heaps of data.
Heaps and heaps and heaps and heaps and heaps and heaps of data.
Heaps and heaps and heaps and heaps and heaps and heaps of data.

Heaps and heaps and heaps and heaps and heaps and heaps of data.
Heaps and heaps and heaps and heaps and heaps and heaps of data.
Heaps and heaps and heaps and heaps and heaps and heaps of data.
Heaps and heaps and heaps and heaps and heaps and heaps of data.
Heaps and heaps and heaps and heaps and heaps and heaps of data.



\end{document}
