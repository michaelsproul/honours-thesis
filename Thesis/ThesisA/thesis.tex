\documentclass[]{unswthesis}

%%% Class options:

%  undergrad (default)
%  hdr

%  11pt (default)
%  12pt

%  final (default)
%  draft

%  oneside (default for hdr)
%  twoside (default for undergrad)


%% Thesis details
\thesistitle{Computer-verified proof of type-soundness for the Linear Language with Locations, $L^3$}
\thesisschool{School of Computer Science and Engineering}
\thesisauthor{Michael Alexander Sproul}
\thesisZid{z3484357}
\thesistopic{} % TODO: Find out what this is meant to be?
\thesisdegree{Bachelor of Science (Honours)}
\thesisdate{October 2015}
\thesissupervisor{Dr. Ben Lippmeier}

%% My own LaTeX macros, definitions, etc
%%%% Shortcuts
\newcommand{\num}[2]{\mbox{#1\,#2}}			% num with units

%%%% Symbols
\newcommand{\yes}{\ensuremath{\surd}\xspace}		% Tick mark
\newcommand{\no}{\ensuremath{\times}\xspace}		% Cross mark
\newcommand{\by}{\ensuremath{\times}\xspace}		% XXX x XXX
\newcommand{\bAND}{\ensuremath{\wedge}\xspace}		% Bool. /\
\newcommand{\bOR}{\ensuremath{\vee}\xspace}		% Bool. \/
\newcommand{\becomes}{\ensuremath{\rightarrow}\xspace}	% -->

%%%% Custom environments

% Centered tabular with single spacing
\newenvironment{ctabular}[1]
    {\par\begin{sspacing}\begin{center}\begin{tabular}{#1}}%
    {\end{tabular}\end{center}\end{sspacing}}


%%%% Our default level for display in TOC - subsubsections
\setcounter{tocdepth}{2}


\begin{document}

%% pages in the ``frontmatter'' section have roman numeral page number
\frontmatter  
\maketitle

\chapter*{Abstract}
\label{abstract}

Write this last.

% \chapter*{Acknowledgements}
\label{ack}

Write this at some point.

% Do I need these?
% \chapter*{Abbreviations}\label{abbr}
\begin{description}
\item[BE] Bachelor of Engineering
\item[EE\&T] School of Electrical Engineering and Telecommunication
\item[\LaTeX] A document preparation computer program
\item[PhD] Doctor of Philosophy
\end{description}


\tableofcontents
%\listoffigures
%\listoftables

%% pages in the ``mainmatter'' section have arabic page numbers and chapters are numbered
\mainmatter

% \chapter{Introduction}\label{ch:intro}

Having a set of clear requirements to their thesis is important to student
finalising their BE, or other, degree.  Such requirements are both in
relation to the physical appearance of the thesis, as well as the writing
style and organisation.  The present document tries to concisely state the
theses requirements while appearing in layout and structure as a thesis
itself.

Chapter~\ref{ch:background} explains the background for this document.
Chapter~\ref{ch:style} states the style and submission related requirements
to theses submitted at the school.
Chapter~\ref{ch:content} explains content related requirements to theses.
Chapter~\ref{ch:eval} evaluates the thesis requirements template.  Finally,
Chapter~\ref{ch:conclusion} draws up conclusions and suggest ways to
further improve the thesis requirements template.


\chapter{Introduction}
\label{ch:intro}

Computer systems form an integral part of modern society, both in the form of personal devices and critical infrastructure. Ensuring the correct operation of computer hardware and software is therefore required. One emerging technique for the construction of robust software systems is the use of mathematical formalisations and proofs of correctness. In this paradigm, desirable properties of the software can be proven true using a computer-based \textit{proof assistant}, which itself relies only on a minimal amount of trusted code. For software formalisation and verification to be truly effective, the objects under consideration must have precise mathematical models associated with them. Typically these models are created based on the \textit{semantics} (meaning) of the programming language that the software is written in. Unfortunately for the would-be software verifier, most popular programming languages lack formal semantics and are therefore not amenable to verification techniques. The focus of this thesis is the computer-based formalisation of language semantics for a specific language (\textit{The Linear Language with Locations} -- $L^3$), as a pre-requisite for further verification of software written in this language.

\section{Operational Semantics}

\section{Type Systems and Type Safety}

\section{Logic, Type Theory and the Coq proof assistant}

\section{Verification Aims}

The $L^3$ language (REF) was specified in a 2001 paper by Ahmed et al. Arguably 

% Background.
\chapter{Background}
\label{ch:intro}

Basic knowledge of

\section{What}

%\chapter{Background}
\label{ch:background}

Etc.

% \include{proposal}
%\chapter{Style and Submission Requirements}\label{ch:style}

Requirements for other parts of the thesis work can be found on the school
web-pages~\cite{Noo05}.  The requirements below are for the written thesis
only.

\section{Format}
The following format specifications must be adhered to for your thesis
(the \LaTeX\ template available from the school ensures this):
\begin{enumerate}
\item The thesis must be printed on \emph{A4 size paper}.
\item The thesis must be typed or prepared using a \emph{word-processor}.
\begin{itemize}
\item For Undergraduate theses, you are encouraged to use both sides
  of the paper.
\item For Higher Degree Research theses, your submitted thesis must be
   printed single-sided.
\end{itemize}
\item \emph{Margins} on all sides must be no less than \unit[25]{mm} (before
binding).
\item \emph{1.5 line spacing} (about \unit[8]{mm} per line) must be used.
\item All sheets must be \emph{numbered}. The main body of the thesis must be
numbered consecutively from beginning to end.  Other sections must either
be included or have their own logical numbering system.
\item The \emph{title page} must contain the following information:
\begin{enumerate}
\item University and School names.
\item Title of Thesis/Project.
\item Topic Number (if applicable).
\item Name of Author and student ID.
\item The degree the thesis is submitted for.
\item Submission date (month and year).
\item Supervisor's name (for undergraduate theses).
\end{enumerate}
\item After the body of the thesis, the thesis \emph{must} contain a
  Bibliography or References list as appropriate.

Authors should confer with their supervisors and School about the
style of their bibliography, as this varies between disciplines.
\end{enumerate}

\section{Other physical appearance}
Other requirements to the physical appearance of your theses are:
\begin{enumerate}
\item The report must be \emph{spiral bound} (at your own cost).
\item Formulas and other items difficult to type may be \emph{neatly
hand-written}
in \emph{permanent} black ink.
\item \emph{Graphs, diagrams and photographs} should be inserted as close as
possible to their \emph{first reference} in the text. Rotated
graphs etc are to be arranged so as to be conveniently read, with the
bottom edge to the outside of the page.
\emph{Graphs and diagrams must be legible!}
\item \emph{Photographs} must be permanently attached to sheets at least along
their left edge. Double sided adhesive may be
used to attach photographs. Photographs printed on A4 size lightweight
paper may be bound directly into the thesis.
\item \emph{Computer programs} and \emph{engineering drawings} should be bound into the
thesis, usually in an appendix.
\item \emph{Floppy diskettes/CD} may be attached to the back cover of the thesis
folder using self adhesive tape or in a secure
pocket.
\end{enumerate}

\section{Submission}

Finally, here are some requirements to the submission procedure. 

\begin{enumerate}
\item The \emph{author} of the thesis is \emph{responsible} for the preparation of the
thesis before the deadline, proofreading the
typescript and having corrections made as necessary.
\item All students must submit a \emph{thesis summary sheet} with their thesis
report. This summary sheet is designed to assist
in determining the overall input by students into the thesis work. Please
note that a separate summary sheet must
be submitted by individual student, even if part of a group submitting a
group thesis.
The guidelines for completing
the summary sheet and the summary sheet form can be downloaded from the
School Office Website.
\item \emph{Two copies} of each thesis/group thesis report must be submitted.
\item Students doing a \emph{Group Thesis} are required to write and hand in
\emph{individual reports}.  The reports should be
clearly distinguishable, and appropriately cross referenced to each other.
The common work overlapping between the reports should be clearly
identified.
\item For undergraduate theses, there is a \emph{page limit} of 100 pages for the main body of the thesis.
\end{enumerate}



\chapter{Content Requirements}\label{ch:content}

Students should consult the literature (e.g.~\cite{Sid99,StrWhi79,Coo64,GRS14})
and other resources for material on how to write a good
thesis.  The present document is only a very brief introduction as to what
is expected.

\nocite{NieLeh03,HasLehKwo05}

\section{Structure}
Most theses are structured very much like the present document.
The main part of the thesis can be structured in many different ways,
however, but must contain: a \emph{problem definition};
\emph{theory} and \emph{considerations} on how to solve the problem;
a description of the \emph{solution method} (dimensioning, construction,
etc.);
presentation of \emph{results} (measurements, simulations, etc.);
a \emph{discussion} of the results (validity, deviations, comparison
with previous solutions, etc.); and finally the \emph{conclusions}.

\section{Style of writing}

\begin{enumerate}

\item Audience:
The thesis must be addressed to engineers at the same level as the
student but without the special knowledge gained during the thesis work.
Such a third-person must be able to reconstruct the results on the basis
of the thesis alone.

\item
Every used concept/symbol/abbreviation which is not widely know must be \emph{defined}.
The wording should be \emph{short} and \emph{concise}.  
For an undergraduate thesis, a suitable length
is 40--70 pages (plus appendices).
Readable(!) \emph{figures} and \emph{graphs} enhances comprehensibility.

\item Units.
\emph{SI units} must be used.
\end{enumerate}

\section{Documentation}

\begin{enumerate}
\item
The work must be well documented; i.e. enclosed must be the \emph{complete
schematics} of designed electronic circuits/test set-ups and/or a
\emph{program listing}, and/or etc.
Documentation of \emph{simulation results} and/or \emph{measurement
results} likewise.
\item References:
For every declaration/equation/method/etc., which is not widely known,
a \emph{reference to the literature} must be given (or a `proof' if it is
the authors own work).
In case material is copied verbatim, quotes must be used.
This is also the case when referring to partners
work in the case of a Group Thesis.

\item Plagiarism:
Failure to give proper references to the literature is \emph{plagiarism}.
Plagiarism is considered serious offence and severe penalties may apply.

\end{enumerate}


%\chapter{Evaluation}\label{ch:eval}

This chapter is mainly provided for the purpose of showing a typical thesis
structure.  There are no more thesis requirements described.

\section{Results}

The result of this work is the present document, being both a \LaTeX\
template and a thesis requirement specification.

\section{Discussion}

The Dual function of this document somewhat de-emphasises the primary
purpose of the document, namely the thesis requirements.  It would be
better, if these could be stated on a few concise pages (cf Appendix
1, p\pageref{app1}).

%\chapter{Conclusion}\label{ch:conclusion}

A thesis requirements/template document has been created.  This serves the
dual purposes of giving students specific requirements to their theses ---
both style and content related --- while providing a typical thesis
structure in a \LaTeX\ template.

\section{Future Work}

Extract the requirements from the template in order to have very concise
requirements.


%% chapters in the ``backmatter'' section do not have chapter numbering
%% text in the ``backmatter'' is single spaced
\backmatter
\bibliographystyle{alpha}
\bibliography{pubs}

%\chapter{Appendix 1}\label{app1}

This section contains the options for the UNSW thesis class; and
layout specifications used by this thesis.

\section{Options}

The standard thesis class options provided are:

\qquad
\begin{tabular}{rl}
undergrad & default \\
hdr & \\[2ex]
11pt & default\\
12pt &\\[2ex]
oneside & default for HDR theses\\
twoside & default for undergraduate theses\\[2ex]
draft & (prints DRAFT on title page and in footer and omits pictures)\\
final & default\\[2ex]
doublespacing & default\\
singlespacing & (only for use while drafting)
\end{tabular}

\section{Margins}

The standard margins for theses in Engineering are as follows:

\qquad
\begin{tabular}{|l|r|r|}
\hline
 & U'grad & HDR\\\hline
{\verb+\oddsidemargin+} & \unit[40]{mm} & \unit[40]{mm}\\
{\verb+\evensidemargin+} & \unit[25]{mm} & \unit[20]{mm}\\
{\verb+\topmargin+} & \unit[25]{mm} & \unit[30]{mm}\\
{\verb+\headheight+} & \unit[40]{mm} & \unit[40]{mm}\\
{\verb+\headsep+} & \unit[40]{mm} & \unit[40]{mm}\\
{\verb+\footskip+} & \unit[15]{mm} & \unit[15]{mm}\\
{\verb+\botmargin+} & \unit[20]{mm} & \unit[20]{mm}\\
\hline
\end{tabular}

\section{Page Headers}

\subsection{Undergraduate Theses}
For undergraduate theses, the page header for odd numbers pages in the
body of the document is:

\quad\fbox{\parbox{.95\textwidth}{Author's Name\hfill \emph{The title of the thesis}}}

and on even pages is:

\quad\fbox{\parbox{.95\textwidth}{\emph{The title of the thesis}\hfill Author's Name}}

These headers are printed on all mainmatter and backmatter pages,
including the first page of chapters or appendices.

\subsection{Higher Degree Research Theses}
For postgraduate theses, the page header for the body of the document is:

\quad\fbox{\parbox{.95\textwidth}{\emph{The title of the chapter or appendix}}}

This header is printed on all mainmatter and backmatter pages,
except for the first page of chapters or appendices.

\section{Page Footers}

For all theses, the page footer consists of a centred page number.  
In the frontmatter, the page number is in roman numerals.  
In the mainmatter and backmatter sections, the page number is in arabic numerals.
Page numbers restart from 1 at the start of the mainmatter section.  

If the \textbf{draft} document option has been selected, then a ``Draft'' message is also inserted into the footer, as in:

\quad\fbox{\parbox{.95\textwidth}{\hfill 14\hfill\hbox to 0pt{\hss\textbf{Draft:} \today}}}

or, on even numbered pages in two-sided mode:

\quad\fbox{\parbox{.95\textwidth}{\leavevmode\hbox to 0pt{\textbf{Draft:} \today\hss}\hfill 14\hfill\mbox{}}}

\section{Double Spacing}
Double spacing (actualy 1.5 spacing) is used for the mainmatter section, except for
footnotes and the text for figures and table.

Single spacing is used in the frontmatter and backmatter sections.

If it is necessary to switch between single-spacing and double-spacing, the commands \verb+\ssp+ and \verb+\dsp+ can be used; or there is a \verb+sspacing+ environment to invoke single spacing and a \verb+spacing+ environment to invoke double spacing if double spacing is used for the document (otherwise it leaves it in single spacing).  Note that switching to single spacing should only be done within the spirit of this thesis class, otherwise it may breach UNSW thesis format guidelines.

\section{Files}

This description and sample of the UNSW Thesis \LaTeX\ class consists of a number of files:

\quad\begin{tabular}{rl}
unswthesis.cls & the thesis class file itself\\[2ex]
crest.pdf & the UNSW coat of arms, used by \verb+pdflatex+ \\
crest.eps & the UNSW coat of arms, used by \verb+latex+ + \verb+dvips+ \\[2ex]
dissertation-sheet.tex & formal information required by HDR theses\\[2ex]
pubs.bib & reference details for use in the bibliography\\[2ex]
sample-thesis.tex & the main file for the thesis
\end{tabular}

The file sample-thesis.tex is the main file for the current document (in use,
its name should be changed to something more meaningful).  It presents
the structure of the thesis, then includes a number of separate files
for the various content sections.  While including separate files is
not essential (it could all be in one file), using multiple files is
useful for organising complex work.

This sample thesis is typical of many theses; however, new authors should
consult with their supervisors and exercise judgement.

The included files used by this sample thesis are:

\quad\begin{tabular}[t]{r}
definitions.tex \\
abstract.tex \\
acknowledgements.tex \\
abbreviations.tex \\
introduction.tex \\
background.tex
\end{tabular}
\quad\begin{tabular}[t]{r}
mywork.tex \\
evaluation.tex \\
conclusion.tex \\
appendix1.tex \\
appendix2.tex 
\end{tabular}

These are typical; however the concepts and names
(and obviously content) of the files making up the matter of the
thesis will differ between theses.

%\chapter{Appendix 2}\label{app2}

This section contains scads of supplimentary data.

\section{Data}

Heaps and heaps and heaps and heaps and heaps and heaps of data.
Heaps and heaps and heaps and heaps and heaps and heaps of data.
Heaps and heaps and heaps and heaps and heaps and heaps of data.
Heaps and heaps and heaps and heaps and heaps and heaps of data.
Heaps and heaps and heaps and heaps and heaps and heaps of data.

Heaps and heaps and heaps and heaps and heaps and heaps of data.
Heaps and heaps and heaps and heaps and heaps and heaps of data.
Heaps and heaps and heaps and heaps and heaps and heaps of data.
Heaps and heaps and heaps and heaps and heaps and heaps of data.
Heaps and heaps and heaps and heaps and heaps and heaps of data.

Heaps and heaps and heaps and heaps and heaps and heaps of data.
Heaps and heaps and heaps and heaps and heaps and heaps of data.
Heaps and heaps and heaps and heaps and heaps and heaps of data.
Heaps and heaps and heaps and heaps and heaps and heaps of data.
Heaps and heaps and heaps and heaps and heaps and heaps of data.

Heaps and heaps and heaps and heaps and heaps and heaps of data.
Heaps and heaps and heaps and heaps and heaps and heaps of data.
Heaps and heaps and heaps and heaps and heaps and heaps of data.
Heaps and heaps and heaps and heaps and heaps and heaps of data.
Heaps and heaps and heaps and heaps and heaps and heaps of data.

Heaps and heaps and heaps and heaps and heaps and heaps of data.
Heaps and heaps and heaps and heaps and heaps and heaps of data.
Heaps and heaps and heaps and heaps and heaps and heaps of data.
Heaps and heaps and heaps and heaps and heaps and heaps of data.
Heaps and heaps and heaps and heaps and heaps and heaps of data.

Heaps and heaps and heaps and heaps and heaps and heaps of data.
Heaps and heaps and heaps and heaps and heaps and heaps of data.
Heaps and heaps and heaps and heaps and heaps and heaps of data.
Heaps and heaps and heaps and heaps and heaps and heaps of data.
Heaps and heaps and heaps and heaps and heaps and heaps of data.

Heaps and heaps and heaps and heaps and heaps and heaps of data.
Heaps and heaps and heaps and heaps and heaps and heaps of data.
Heaps and heaps and heaps and heaps and heaps and heaps of data.
Heaps and heaps and heaps and heaps and heaps and heaps of data.
Heaps and heaps and heaps and heaps and heaps and heaps of data.

Heaps and heaps and heaps and heaps and heaps and heaps of data.
Heaps and heaps and heaps and heaps and heaps and heaps of data.
Heaps and heaps and heaps and heaps and heaps and heaps of data.
Heaps and heaps and heaps and heaps and heaps and heaps of data.
Heaps and heaps and heaps and heaps and heaps and heaps of data.

Heaps and heaps and heaps and heaps and heaps and heaps of data.
Heaps and heaps and heaps and heaps and heaps and heaps of data.
Heaps and heaps and heaps and heaps and heaps and heaps of data.
Heaps and heaps and heaps and heaps and heaps and heaps of data.
Heaps and heaps and heaps and heaps and heaps and heaps of data.

Heaps and heaps and heaps and heaps and heaps and heaps of data.
Heaps and heaps and heaps and heaps and heaps and heaps of data.
Heaps and heaps and heaps and heaps and heaps and heaps of data.
Heaps and heaps and heaps and heaps and heaps and heaps of data.
Heaps and heaps and heaps and heaps and heaps and heaps of data.

Heaps and heaps and heaps and heaps and heaps and heaps of data.
Heaps and heaps and heaps and heaps and heaps and heaps of data.
Heaps and heaps and heaps and heaps and heaps and heaps of data.
Heaps and heaps and heaps and heaps and heaps and heaps of data.
Heaps and heaps and heaps and heaps and heaps and heaps of data.



\end{document}
