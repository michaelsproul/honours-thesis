\documentclass[]{unswthesis}

\usepackage{hyperref}

% Disable hideous fluorescent links
\hypersetup{hidelinks = true}

%%% Class options:

%  undergrad (default)
%  hdr

%  11pt (default)
%  12pt

%  final (default)
%  draft

%  oneside (default for hdr)
%  twoside (default for undergrad)


%% Thesis details
\thesistitle{Computer-verified proof of type-soundness for the Linear Language with Locations, $L^3$}
\thesisschool{School of Computer Science and Engineering}
\thesisauthor{Michael Alexander Sproul}
\thesisZid{z3484357}
\thesistopic{} % TODO: Find out what this is meant to be?
\thesisdegree{Bachelor of Science (Honours)}
\thesisdate{October 2015}
\thesissupervisor{Dr. Ben Lippmeier}

%% My own LaTeX macros, definitions, etc
\include{definitions}

\begin{document}

%% pages in the ``frontmatter'' section have roman numeral page number
\frontmatter  
\maketitle

% \chapter*{Abstract}
\label{abstract}

Write this last.

% \chapter*{Acknowledgements}
\label{ack}

Write this at some point.

% Do I need these?
% \include{abbreviations}

\tableofcontents
%\listoffigures
%\listoftables

%% pages in the ``mainmatter'' section have arabic page numbers and chapters are numbered
\mainmatter

% \include{introduction}
\chapter{Introduction}
\label{ch:intro}

Computer systems form an integral part of modern society, both in the form of personal devices and critical infrastructure. Ensuring the correct operation of computer hardware and software is therefore required. One emerging technique for the construction of robust software systems is the use of mathematical formalisations and proofs of correctness. In this paradigm, desirable properties of the software can be proven true using a computer-based \textit{proof assistant}, which itself relies only on a minimal amount of trusted code. For software formalisation and verification to be truly effective, the objects under consideration must have precise mathematical models associated with them. Typically these models are created based on the \textit{semantics} (meaning) of the programming language that the software is written in. Unfortunately for the would-be software verifier, most popular programming languages lack formal semantics and are therefore not amenable to verification techniques. The focus of this thesis is the computer-based formalisation of language semantics for a specific language (\textit{The Linear Language with Locations} -- $L^3$), as a pre-requisite for further verification of software written in this language.

\section{Operational Semantics}

\section{Type Systems and Type Safety}

\section{Logic, Type Theory and the Coq proof assistant}

\section{Verification Aims}

The $L^3$ language was specified in a 2001 paper by Ahmed et al (REF). The paper includes a hand-written proof of type soundness for $L^3$ core, spanning 8 pages.

% Background.
\chapter{Background}
\label{ch:intro}

\section{Previous Work}

\subsection{Linear and Uniqueness Typing}

de Vries. Cyclone. Clean.

\subsection{Systems of Capabilities}

Mezzo. Strong updates, etc.

\subsection{Trust-worthy compilers, typed assembly languages and other similar systems}

\section{The Linear Language with Locations, $L^3$}

\chapter{Proposal}

\section{Variable naming and binding}

One problem that arises frequently in the formalisation of language semantics is that of \textit{capture-avoiding substitution}. Substitution operations, whereby a value is substituted for a variable in a term, form the core computational component of the operational semantics in many languages. In the simply-typed (and untyped) lambda calculus, the $\beta$-rule uses substitution (denoted $e[v/x]$) to describe the semantics of function application:
% FIXME: formatting, long right arrow and space.
\begin{eqnarray*}
(\lambda x : \tau. e) v \Rightarrow_\beta e[v/x]
\end{eqnarray*}

The problem of \textit{variable capture}, which we wish to avoid, is demonstrated by the following example:
% FIXME: formatting
\begin{eqnarray*}
(\lambda x. \lambda y. x + y) y \Rightarrow_\beta (\lambda y. y + y)
\end{eqnarray*}

Here the parameter $y$ is a free variable acting as a place-holder for a value in the environment. After substitution however, the $y$ replacing $x$ in the abstraction body $x + y$ becomes bound due to the name collision between the free $y$ and the binder $y$. Intuitively, drastically altering the meaning of terms during substitution is something we would like to avoid.

One way to avoid variable capture is to forbid the substitution of any terms containing free variables. In such a system, free variables like $y$ are never considered values and as such cannot be used in (variable capturing) substitutions. This is the approach taken by \textit{Software Foundations} (REF) in formalisations of the simply-typed lambda calculus and its variants. A further consequence of this approach is that globally-shared integers or strings for variable names are sufficient to guarantee soundness. Although it's tempting to embrace this approach for its simplifying properties, it doesn't accurately capture the behaviour of common functional languages like Haskell and ML, which perform substitutions whilst avoiding variable capture. (might need a stronger argument here).

In our Coq formalisation of $L^3$ we would therefore like to include \textit{capture avoiding substitution} as part of the definitions of variable names and substitution operations. For this we consider three main approaches from the literature which all exploit the observation that the exact names of bound variables are insignificant at the level of language formalisation. In other words, although the names of variables may hold meaning for the authors of programs, they do not impact the meaning of the programs themselves.

\subsection{Higher-order Abstract Syntax}

\subsection{de Bruijn indices}

\subsection{The Locally Nameless approach}



%\chapter{Background}
\label{ch:background}

Etc.

%\include{proposal}
%\include{mywork}
%\include{evaluation}
%\include{conclusion}

%% chapters in the ``backmatter'' section do not have chapter numbering
%% text in the ``backmatter'' is single spaced
\backmatter
\bibliographystyle{alpha}
\bibliography{pubs}

%\include{appendix1}
%\include{appendix2}

\end{document}
